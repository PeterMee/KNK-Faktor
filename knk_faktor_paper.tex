\documentclass[12pt,a4paper]{article}

% Packages
\usepackage[T1]{fontenc}
\usepackage[utf8]{inputenc}
\usepackage{lmodern}
\usepackage{microtype}
\usepackage{graphicx}
\usepackage{amsmath, amssymb}
\usepackage{geometry}
\usepackage{setspace}
\usepackage{hyperref}
\usepackage{csquotes}
\usepackage{booktabs}

% Page setup
\geometry{margin=1in}
\onehalfspacing

% Title
\title{\textbf{Der Kosten-Nutzen-Kopfweh-Faktor}\\[0.5em]
\large Ein ganzheitliches Modell zur Bewertung von Aufwand, Nutzen und subjektiver Belastung}

\author{Peter Meyer}
\date{\today}

\begin{document}
\maketitle

\begin{abstract}
In Entscheidungsprozessen, sowohl im beruflichen als auch im privaten Kontext, dominieren häufig rationale Abwägungen von Kosten und Nutzen. Es wird geschätzt, kalkuliert und optimiert – stets mit dem Ziel, die effizienteste Option zu identifizieren. Doch diese scheinbar objektive Betrachtung blendet einen entscheidenden Faktor aus: den subjektiv empfundenen Aufwand, der sich nicht in Zahlenwerken ausdrücken lässt. Viele Vorhaben verursachen unverhältnismäßig viel „Kopfweh“ – Stress, Frustration, Unsicherheit, mentale Belastung oder schlicht organisatorisches Chaos, das sich in klassischen Kosten-Nutzen-Modellen nicht wiederfindet, obwohl es maßgeblich bestimmt, wie sinnvoll, machbar oder nachhaltig ein Projekt tatsächlich ist.
\bigskip
Diese Lücke führt dazu, dass Entscheidungen oft systematisch verzerrt getroffen werden. Tätigkeiten, die theoretisch effizient erscheinen, entpuppen sich in der Praxis als hochgradig belastend, während scheinbar „teure“ Alternativen subjektiv deutlich leichter zu bewältigen sind. Um diese Diskrepanz abzubilden, wird in diesem Paper der Kosten-Nutzen-Kopfweh-Faktor (KNK-Faktor) eingeführt: ein Modell, das Nutzen, objektiven Aufwand und subjektive Belastung in einem einzigen, interpretierbaren Indikator zusammenführt.
\bigskip
Der KNK-Faktor soll es ermöglichen, Entscheidungen realitätsnäher zu bewerten, Stressoren explizit zu quantifizieren und damit eine Brücke zwischen ökonomischen Ansätzen und psychologischen Erkenntnissen über Belastung, kognitive Beanspruchung und Entscheidungsverhalten zu schlagen. Ziel dieses Papers ist es, den theoretischen Rahmen des KNK-Faktors zu entwickeln, eine praktikable Operationalisierung vorzuschlagen und potenzielle Anwendungsszenarien in Forschung und Praxis aufzuzeigen.

\end{abstract}

\newpage

\section{Einleitung}
Einführung in das Thema, Motivation, Problemstellung, Zielsetzung des Papers.

\section{Verwandte Arbeiten}
Übersicht über existierende Forschung, relevante Modelle oder theoretische Grundlagen.

\section{Methodik}
Beschreibe dein Modell, deine Herangehensweise, Datengrundlage oder theoretische Ableitungen.

\section{Ergebnisse}
Präsentiere deine wichtigsten Ergebnisse, Modelle, Berechnungen oder Visualisierungen.

\section{Diskussion}
Bewerte und interpretiere deine Ergebnisse. Was bedeuten sie? Wo liegen Grenzen?

\section{Fazit}
Schließe dein Paper ab: wichtigste Erkenntnisse, Ausblick, mögliche zukünftige Arbeiten.

\bibliographystyle{plain}
\bibliography{references}

\end{document}
