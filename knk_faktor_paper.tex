\documentclass[12pt,a4paper]{article}

% Packages
\usepackage[T1]{fontenc}
\usepackage[utf8]{inputenc}
\usepackage{lmodern}
\usepackage{microtype}
\usepackage{graphicx}
\usepackage{amsmath, amssymb}
\usepackage{geometry}
\usepackage{setspace}
\usepackage{hyperref}
\usepackage{csquotes}
\usepackage{booktabs}

% Page setup
\geometry{margin=1in}
\onehalfspacing

% Title
\title{\textbf{Der Kosten-Nutzen-Kopfweh-Faktor}\\[0.5em]
\large Ein ganzheitliches Modell zur Bewertung von Aufwand, Nutzen und subjektiver Belastung}

\author{Peter Meyer}
\date{\today}

\begin{document}
\maketitle

\begin{abstract}
In Entscheidungsprozessen, sowohl im beruflichen als auch im privaten Kontext, dominieren häufig rationale Abwägungen von Kosten und Nutzen. Es wird geschätzt, kalkuliert und optimiert – stets mit dem Ziel, die effizienteste Option zu identifizieren. Doch diese scheinbar objektive Betrachtung blendet einen entscheidenden Faktor aus: den subjektiv empfundenen Aufwand, der sich nicht in Zahlenwerken ausdrücken lässt. Viele Vorhaben verursachen unverhältnismäßig viel „Kopfweh“ – Stress, Frustration, Unsicherheit, mentale Belastung oder schlicht organisatorisches Chaos, das sich in klassischen Kosten-Nutzen-Modellen nicht wiederfindet, obwohl es maßgeblich bestimmt, wie sinnvoll, machbar oder nachhaltig ein Projekt tatsächlich ist.
\bigskip
Diese Lücke führt dazu, dass Entscheidungen oft systematisch verzerrt getroffen werden. Tätigkeiten, die theoretisch effizient erscheinen, entpuppen sich in der Praxis als hochgradig belastend, während scheinbar „teure“ Alternativen subjektiv deutlich leichter zu bewältigen sind. Um diese Diskrepanz abzubilden, wird in diesem Paper der Kosten-Nutzen-Kopfweh-Faktor (KNK-Faktor) eingeführt: ein Modell, das Nutzen, objektiven Aufwand und subjektive Belastung in einem einzigen, interpretierbaren Indikator zusammenführt.
\bigskip
Der KNK-Faktor soll es ermöglichen, Entscheidungen realitätsnäher zu bewerten, Stressoren explizit zu quantifizieren und damit eine Brücke zwischen ökonomischen Ansätzen und psychologischen Erkenntnissen über Belastung, kognitive Beanspruchung und Entscheidungsverhalten zu schlagen. Ziel dieses Papers ist es, den theoretischen Rahmen des KNK-Faktors zu entwickeln, eine praktikable Operationalisierung vorzuschlagen und potenzielle Anwendungsszenarien in Forschung und Praxis aufzuzeigen.

\end{abstract}



\newpage
\section{Mathematische Herleitung des KNK-Faktors}
Klassische Kosten-Nutzen-Modelle bewerten Entscheidungen primär anhand des Verhältnisses zwischen erwarteten Erträgen und objektiven Kosten. Diese Modelle vernachlässigen jedoch Belastungsfaktoren wie Stress, kognitiver Beanspruchung oder emotionaler Frustration, die in der Praxis einen relevanten Einfluss auf die Durchführbarkeit und Sinnhaftigkeit eines Vorhabens ausüben. Um diese Lücke zu schließen, wird im Folgenden der Kosten-Nutzen-Kopfweh-Faktor (KNK) als erweiterter Effizienzindikator hergeleitet, der sowohl objektive als auch subjektive Belastungskomponenten integriert.

\subsection{Erweiterung der Kostenkomponente um subjektive Belastungskomponente}
Ausgangspunkt ist das klassische Kosten-Nutzen-Verhältnis:
\[
\text{$E$} = \frac{\text{$N$}}{\text{$C$}},
\]
wobei $N$ den erwarteten Nutzen und $C$ die Gesamtkosten repräsentiert.
Im Kontext alltäglicher oder beruflicher Entscheidungen besteht der Kostenbegriff jedoch nicht allein aus quantifizierbarem Aufwand (z. B. Zeit, Geld, Ressourcen), sondern umfasst zusätzlich psychologische Belastungsfaktoren, die den realen Aufwand subjektiv erhöhen. Diese Belastungen werden im Modell als Kopfweh-Komponente 
$K$ bezeichnet.
Die Gesamtkosten eines Vorhabens lassen sich somit konzeptuell als additive Struktur modellieren:
\[
C_{\text{gesamt}} = \text{$A$} + \text{$K$},
\]
wobei $A$ den objektiven Aufwand darstellt und $K$ die subjektive Belastung repräsentiert.
Die Additivität ist hier die naheliegendste und einfachste Annahme: Sowohl Aufwand als auch Belastung wirken unabhängige, kumulative Kostenbeiträge, die gemeinsam die „Gesamtbelastung“ einer Aufgabe definieren.

\subsection{Normalisierung der Variablen}
Da Aufwand, Nutzen und Kopfweh typischerweise auf unterschiedlichen Skalen vorliegen (z. B. Stunden, Euro, psychologische Ratings), ist eine Transformation notwendig, um die Variablen vergleichbar zu machen. Jede Variable 
$X \in \{N, A, K\}$ wird daher auf eine einheitliche Skala zwischen 0 und 1 normalisiert:
\[
\text{$X$’} = \frac{\text{$X$ - min($X$)}}{\text{max($X$ - min($X$)}}.
\]
Dies gewährleistet, dass keine Variable allein aufgrund ihrer Messskala unverhältnismäßig starken Einfluss auf das Modell ausübt.


%\bigskip
%TODO beispiel für normalisierung 

\subsection{Einführung von Gewichtungsfaktoren}
Nicht jede Komponente trägt in jeder Entscheidung den gleichen Anteil zur Gesamteffizienz bei. Projektabhängig kann Nutzen wichtiger sein als Kopfweh, oder Belastung kann schwerer wiegen als objektiver Aufwand. Daher werden Gewichtungsfaktoren eingeführt:

\begin{itemize}
    \item $w_A \cdot A$ \; (Gewichtung $\times$ Aufwand)
    \item $w_K \cdot K$ \; (Gewichtung $\times$ Kopfweh / subjektive Belastung)
    \item $w_N \cdot N$ \; (Gewichtung $\times$ Nutzen)
\end{itemize}
Diese Gewichte ermöglichen die Anpassung des Modells an unterschiedliche Kontexte (z. B. Unternehmensentscheidungen, persönliche Lebensentscheidungen, medizinische Abläufe). 
Die gewichtete Nutzenkomponente lautet somit $w_N N$', die gewichteten Gesamtkosten sind $w_A A$' + $w_K K$'.

\subsection{Regularisierung zur Vermeidung von Division durch Null}
Falls sowohl Aufwand als auch Kopfweh nach Normalisierung den Wert 0 annehmen, wäre der Nenner der Formel gleich null und die Division undefiniert. Darüber hinaus führt ein sehr kleiner Nenner zu extrem großen, empirisch instabilen KNK-Werten. Um diese Probleme zu umgehen, wird ein kleiner Regularisierungsterm 
$\epsilon$ >0 addiert:
\[
C_{\text{eff}} = w_A\text{$A$'} + w_K\text{$K$'} + \epsilon.
\]
Typische Werte für $\epsilon$ liegen zwischen 0.01 und 0.1, abhängig vom Skalenniveau.

\subsection{Definition des KNK-Faktors}
Auf Basis der oben genannten Komponenten ergibt sich der Kosten-Nutzen-Kopfweh-Faktor (KNK) als:
\begin{center}
    \boxed{\text{KNK} = \frac{w_N\text{$N$'}}{w_A\text{$A$'} + w_K\text{$K$'} + \epsilon}}
\end{center}
Dieser Quotient repräsentiert den Nutzen pro Einheit effektiver Belastung.
Ein hoher KNK-Wert zeigt an, dass ein Vorhaben einen vergleichsweise hohen Nutzen bei niedrigem Aufwand und geringer subjektiver Belastung bietet. Ein niedriger KNK-Wert hingegen signalisiert, dass Aufwand und Kopfweh im Verhältnis zum Nutzen überwiegen.




\newpage
\[
\text{KNK-Faktor} = \frac{\text{Nutzen}}{\text{Aufwand} + \text{Belastung}}
\]

Gewichtete Formel mit: \\
$w_A \cdot A$ \; (Gewichtung $\times$ Aufwand) \\
$w_K \cdot K$ \; (Gewichtung $\times$ Kopfweh / subjektive Belastung) \\
$w_N \cdot N$ \; (Gewichtung $\times$ Nutzen) \\




\[
\text{KNK} = \frac{w_N \cdot N}{w_A \cdot A + w_K \cdot K}
\]

\newpage




\section{Einleitung}
Einführung in das Thema, Motivation, Problemstellung, Zielsetzung des Papers.

\section{Verwandte Arbeiten}
Übersicht über existierende Forschung, relevante Modelle oder theoretische Grundlagen.

\section{Methodik}
Beschreibe dein Modell, deine Herangehensweise, Datengrundlage oder theoretische Ableitungen.

\section{Ergebnisse}
Präsentiere deine wichtigsten Ergebnisse, Modelle, Berechnungen oder Visualisierungen.

\section{Diskussion}
Bewerte und interpretiere deine Ergebnisse. Was bedeuten sie? Wo liegen Grenzen?

\section{Fazit}
Schließe dein Paper ab: wichtigste Erkenntnisse, Ausblick, mögliche zukünftige Arbeiten.

\bibliographystyle{plain}
\bibliography{references}

\end{document}
